% INTRODUCTION
\chapter{Introduction}

\section{About the course}

The course \emph{Multimedia: modelleren \& pogrammeren} is given by prof. E. Duval at KULeuven. The objective of this course to create a mobile application using different technologies, namely \texttt{iOS}, \texttt{Android} and \texttt{HTML5}, to eventually gain insight in the advantages and disadvantages of each technology.

The technologies that will be discussed on this blog are \texttt{iOS} and \texttt{Android}. All code for this project is open source and can be found on and downloaded from Github\footnote{\url{https://github.com/mumedev}}. The progress of the development is communicated through a blog\footnote{\url{http://mumedev.wordpress.com/}}.


\section{About the application}

The application is an online forum where the main question of each tread is directed at one expert or a group of experts. Each member can enlist him/herself as an expert in certain areas and will get notified when a question is posted, related to his/her domain of expertise.

%Badges can be obtained (providing useful comments for a number of problems), and certificates (having an actual degree in a certain area)
%if specific for KULeuven --> credits obtained for certain courses ~ expertise

Although similar websites already exist, e.g. \emph{Stackoverflow}\footnote{\url{http://stackoverflow.com/}} and \emph{Yahoo! Answers}\footnote{\url{http://answers.yahoo.com/}}, one idea would be to direct the application rather at students than at the general public. The idea arose when looking at the rather limited use of the fora on \emph{Toledo}\footnote{\url{https://toledo.kuleuven.be}}.


\section{Structure of this text}

The next chapter looks at the idea behind the application in more detail. We use tools such as user stories, story boards, use case diagrams, and screen transition diagrams to get an understanding of the application from a user's perspective. From these functional requirements, a number of general requirements for the software can be derived.

These requirements serve as the input for chapters \ref{chapter:architecture} and \ref{chapter:implementation}. In these chapter we will determine how the architecture of the application is constructed on three levels: a global scope, the data itself, and each app individually.

Chapter \ref{chapter:comparison} looks at different aspects of each mobile technology and tries to compare them.

Finally chapter \ref{chapter:conclusion} looks back at the project, points presented in the discussion, and the course as a whole.


