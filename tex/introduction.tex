% INTRODUCTION
\chapter{Introduction}

\section{About this report}

The course \emph{Multimedia: modelleren \& pogrammeren} is given by prof. E. Duval at KULeuven. The objective of this course to create a mobile application using different technologies, namely \texttt{iOS}, \texttt{Android} and \texttt{HTML5}, to eventually gain insight in the advantages and disadvantages of each technology.

The technologies that will be discussed on this blog are \texttt{iOS} and \texttt{Android}. All code for this project is open source and can be found on and downloaded from Github\footnote{\url{https://github.com/mumedev}}. The progress of the development is communicated through a blog.


\section{About the application}

The application is an online forum where the main question of each tread is directed at one expert or a group of experts. Each member can enlist him/herself as an expert in certain areas and will get notified when a question is posted, related to his/her domain of expertise.

%Badges can be obtained (providing useful comments for a number of problems), and certificates (having an actual degree in a certain area)
%if specific for KULeuven --> credits obtained for certain courses ~ expertise

Although similar websites already exist, e.g. \emph{Stackoverflow}\footnote{\url{http://stackoverflow.com/}} and \emph{Yahoo! Answers}\footnote{\url{http://answers.yahoo.com/}}, one idea would be to direct the application rather at students than at the general public.
